
% Default to the notebook output style

    


% Inherit from the specified cell style.




    
\documentclass[11pt]{article}

    
    
    \usepackage[T1]{fontenc}
    % Nicer default font (+ math font) than Computer Modern for most use cases
    \usepackage{mathpazo}

    % Basic figure setup, for now with no caption control since it's done
    % automatically by Pandoc (which extracts ![](path) syntax from Markdown).
    \usepackage{graphicx}
    % We will generate all images so they have a width \maxwidth. This means
    % that they will get their normal width if they fit onto the page, but
    % are scaled down if they would overflow the margins.
    \makeatletter
    \def\maxwidth{\ifdim\Gin@nat@width>\linewidth\linewidth
    \else\Gin@nat@width\fi}
    \makeatother
    \let\Oldincludegraphics\includegraphics
    % Set max figure width to be 80% of text width, for now hardcoded.
    \renewcommand{\includegraphics}[1]{\Oldincludegraphics[width=.8\maxwidth]{#1}}
    % Ensure that by default, figures have no caption (until we provide a
    % proper Figure object with a Caption API and a way to capture that
    % in the conversion process - todo).
    \usepackage{caption}
    \DeclareCaptionLabelFormat{nolabel}{}
    \captionsetup{labelformat=nolabel}

    \usepackage{adjustbox} % Used to constrain images to a maximum size 
    \usepackage{xcolor} % Allow colors to be defined
    \usepackage{enumerate} % Needed for markdown enumerations to work
    \usepackage{geometry} % Used to adjust the document margins
    \usepackage{amsmath} % Equations
    \usepackage{amssymb} % Equations
    \usepackage{textcomp} % defines textquotesingle
    % Hack from http://tex.stackexchange.com/a/47451/13684:
    \AtBeginDocument{%
        \def\PYZsq{\textquotesingle}% Upright quotes in Pygmentized code
    }
    \usepackage{upquote} % Upright quotes for verbatim code
    \usepackage{eurosym} % defines \euro
    \usepackage[mathletters]{ucs} % Extended unicode (utf-8) support
    \usepackage[utf8x]{inputenc} % Allow utf-8 characters in the tex document
    \usepackage{fancyvrb} % verbatim replacement that allows latex
    \usepackage{grffile} % extends the file name processing of package graphics 
                         % to support a larger range 
    % The hyperref package gives us a pdf with properly built
    % internal navigation ('pdf bookmarks' for the table of contents,
    % internal cross-reference links, web links for URLs, etc.)
    \usepackage{hyperref}
    \usepackage{longtable} % longtable support required by pandoc >1.10
    \usepackage{booktabs}  % table support for pandoc > 1.12.2
    \usepackage[inline]{enumitem} % IRkernel/repr support (it uses the enumerate* environment)
    \usepackage[normalem]{ulem} % ulem is needed to support strikethroughs (\sout)
                                % normalem makes italics be italics, not underlines
    

    
    
    % Colors for the hyperref package
    \definecolor{urlcolor}{rgb}{0,.145,.698}
    \definecolor{linkcolor}{rgb}{.71,0.21,0.01}
    \definecolor{citecolor}{rgb}{.12,.54,.11}

    % ANSI colors
    \definecolor{ansi-black}{HTML}{3E424D}
    \definecolor{ansi-black-intense}{HTML}{282C36}
    \definecolor{ansi-red}{HTML}{E75C58}
    \definecolor{ansi-red-intense}{HTML}{B22B31}
    \definecolor{ansi-green}{HTML}{00A250}
    \definecolor{ansi-green-intense}{HTML}{007427}
    \definecolor{ansi-yellow}{HTML}{DDB62B}
    \definecolor{ansi-yellow-intense}{HTML}{B27D12}
    \definecolor{ansi-blue}{HTML}{208FFB}
    \definecolor{ansi-blue-intense}{HTML}{0065CA}
    \definecolor{ansi-magenta}{HTML}{D160C4}
    \definecolor{ansi-magenta-intense}{HTML}{A03196}
    \definecolor{ansi-cyan}{HTML}{60C6C8}
    \definecolor{ansi-cyan-intense}{HTML}{258F8F}
    \definecolor{ansi-white}{HTML}{C5C1B4}
    \definecolor{ansi-white-intense}{HTML}{A1A6B2}

    % commands and environments needed by pandoc snippets
    % extracted from the output of `pandoc -s`
    \providecommand{\tightlist}{%
      \setlength{\itemsep}{0pt}\setlength{\parskip}{0pt}}
    \DefineVerbatimEnvironment{Highlighting}{Verbatim}{commandchars=\\\{\}}
    % Add ',fontsize=\small' for more characters per line
    \newenvironment{Shaded}{}{}
    \newcommand{\KeywordTok}[1]{\textcolor[rgb]{0.00,0.44,0.13}{\textbf{{#1}}}}
    \newcommand{\DataTypeTok}[1]{\textcolor[rgb]{0.56,0.13,0.00}{{#1}}}
    \newcommand{\DecValTok}[1]{\textcolor[rgb]{0.25,0.63,0.44}{{#1}}}
    \newcommand{\BaseNTok}[1]{\textcolor[rgb]{0.25,0.63,0.44}{{#1}}}
    \newcommand{\FloatTok}[1]{\textcolor[rgb]{0.25,0.63,0.44}{{#1}}}
    \newcommand{\CharTok}[1]{\textcolor[rgb]{0.25,0.44,0.63}{{#1}}}
    \newcommand{\StringTok}[1]{\textcolor[rgb]{0.25,0.44,0.63}{{#1}}}
    \newcommand{\CommentTok}[1]{\textcolor[rgb]{0.38,0.63,0.69}{\textit{{#1}}}}
    \newcommand{\OtherTok}[1]{\textcolor[rgb]{0.00,0.44,0.13}{{#1}}}
    \newcommand{\AlertTok}[1]{\textcolor[rgb]{1.00,0.00,0.00}{\textbf{{#1}}}}
    \newcommand{\FunctionTok}[1]{\textcolor[rgb]{0.02,0.16,0.49}{{#1}}}
    \newcommand{\RegionMarkerTok}[1]{{#1}}
    \newcommand{\ErrorTok}[1]{\textcolor[rgb]{1.00,0.00,0.00}{\textbf{{#1}}}}
    \newcommand{\NormalTok}[1]{{#1}}
    
    % Additional commands for more recent versions of Pandoc
    \newcommand{\ConstantTok}[1]{\textcolor[rgb]{0.53,0.00,0.00}{{#1}}}
    \newcommand{\SpecialCharTok}[1]{\textcolor[rgb]{0.25,0.44,0.63}{{#1}}}
    \newcommand{\VerbatimStringTok}[1]{\textcolor[rgb]{0.25,0.44,0.63}{{#1}}}
    \newcommand{\SpecialStringTok}[1]{\textcolor[rgb]{0.73,0.40,0.53}{{#1}}}
    \newcommand{\ImportTok}[1]{{#1}}
    \newcommand{\DocumentationTok}[1]{\textcolor[rgb]{0.73,0.13,0.13}{\textit{{#1}}}}
    \newcommand{\AnnotationTok}[1]{\textcolor[rgb]{0.38,0.63,0.69}{\textbf{\textit{{#1}}}}}
    \newcommand{\CommentVarTok}[1]{\textcolor[rgb]{0.38,0.63,0.69}{\textbf{\textit{{#1}}}}}
    \newcommand{\VariableTok}[1]{\textcolor[rgb]{0.10,0.09,0.49}{{#1}}}
    \newcommand{\ControlFlowTok}[1]{\textcolor[rgb]{0.00,0.44,0.13}{\textbf{{#1}}}}
    \newcommand{\OperatorTok}[1]{\textcolor[rgb]{0.40,0.40,0.40}{{#1}}}
    \newcommand{\BuiltInTok}[1]{{#1}}
    \newcommand{\ExtensionTok}[1]{{#1}}
    \newcommand{\PreprocessorTok}[1]{\textcolor[rgb]{0.74,0.48,0.00}{{#1}}}
    \newcommand{\AttributeTok}[1]{\textcolor[rgb]{0.49,0.56,0.16}{{#1}}}
    \newcommand{\InformationTok}[1]{\textcolor[rgb]{0.38,0.63,0.69}{\textbf{\textit{{#1}}}}}
    \newcommand{\WarningTok}[1]{\textcolor[rgb]{0.38,0.63,0.69}{\textbf{\textit{{#1}}}}}
    
    
    % Define a nice break command that doesn't care if a line doesn't already
    % exist.
    \def\br{\hspace*{\fill} \\* }
    % Math Jax compatability definitions
    \def\gt{>}
    \def\lt{<}
    % Document parameters
    \title{Symbolics}
    
    
    

    % Pygments definitions
    
\makeatletter
\def\PY@reset{\let\PY@it=\relax \let\PY@bf=\relax%
    \let\PY@ul=\relax \let\PY@tc=\relax%
    \let\PY@bc=\relax \let\PY@ff=\relax}
\def\PY@tok#1{\csname PY@tok@#1\endcsname}
\def\PY@toks#1+{\ifx\relax#1\empty\else%
    \PY@tok{#1}\expandafter\PY@toks\fi}
\def\PY@do#1{\PY@bc{\PY@tc{\PY@ul{%
    \PY@it{\PY@bf{\PY@ff{#1}}}}}}}
\def\PY#1#2{\PY@reset\PY@toks#1+\relax+\PY@do{#2}}

\expandafter\def\csname PY@tok@w\endcsname{\def\PY@tc##1{\textcolor[rgb]{0.73,0.73,0.73}{##1}}}
\expandafter\def\csname PY@tok@c\endcsname{\let\PY@it=\textit\def\PY@tc##1{\textcolor[rgb]{0.25,0.50,0.50}{##1}}}
\expandafter\def\csname PY@tok@cp\endcsname{\def\PY@tc##1{\textcolor[rgb]{0.74,0.48,0.00}{##1}}}
\expandafter\def\csname PY@tok@k\endcsname{\let\PY@bf=\textbf\def\PY@tc##1{\textcolor[rgb]{0.00,0.50,0.00}{##1}}}
\expandafter\def\csname PY@tok@kp\endcsname{\def\PY@tc##1{\textcolor[rgb]{0.00,0.50,0.00}{##1}}}
\expandafter\def\csname PY@tok@kt\endcsname{\def\PY@tc##1{\textcolor[rgb]{0.69,0.00,0.25}{##1}}}
\expandafter\def\csname PY@tok@o\endcsname{\def\PY@tc##1{\textcolor[rgb]{0.40,0.40,0.40}{##1}}}
\expandafter\def\csname PY@tok@ow\endcsname{\let\PY@bf=\textbf\def\PY@tc##1{\textcolor[rgb]{0.67,0.13,1.00}{##1}}}
\expandafter\def\csname PY@tok@nb\endcsname{\def\PY@tc##1{\textcolor[rgb]{0.00,0.50,0.00}{##1}}}
\expandafter\def\csname PY@tok@nf\endcsname{\def\PY@tc##1{\textcolor[rgb]{0.00,0.00,1.00}{##1}}}
\expandafter\def\csname PY@tok@nc\endcsname{\let\PY@bf=\textbf\def\PY@tc##1{\textcolor[rgb]{0.00,0.00,1.00}{##1}}}
\expandafter\def\csname PY@tok@nn\endcsname{\let\PY@bf=\textbf\def\PY@tc##1{\textcolor[rgb]{0.00,0.00,1.00}{##1}}}
\expandafter\def\csname PY@tok@ne\endcsname{\let\PY@bf=\textbf\def\PY@tc##1{\textcolor[rgb]{0.82,0.25,0.23}{##1}}}
\expandafter\def\csname PY@tok@nv\endcsname{\def\PY@tc##1{\textcolor[rgb]{0.10,0.09,0.49}{##1}}}
\expandafter\def\csname PY@tok@no\endcsname{\def\PY@tc##1{\textcolor[rgb]{0.53,0.00,0.00}{##1}}}
\expandafter\def\csname PY@tok@nl\endcsname{\def\PY@tc##1{\textcolor[rgb]{0.63,0.63,0.00}{##1}}}
\expandafter\def\csname PY@tok@ni\endcsname{\let\PY@bf=\textbf\def\PY@tc##1{\textcolor[rgb]{0.60,0.60,0.60}{##1}}}
\expandafter\def\csname PY@tok@na\endcsname{\def\PY@tc##1{\textcolor[rgb]{0.49,0.56,0.16}{##1}}}
\expandafter\def\csname PY@tok@nt\endcsname{\let\PY@bf=\textbf\def\PY@tc##1{\textcolor[rgb]{0.00,0.50,0.00}{##1}}}
\expandafter\def\csname PY@tok@nd\endcsname{\def\PY@tc##1{\textcolor[rgb]{0.67,0.13,1.00}{##1}}}
\expandafter\def\csname PY@tok@s\endcsname{\def\PY@tc##1{\textcolor[rgb]{0.73,0.13,0.13}{##1}}}
\expandafter\def\csname PY@tok@sd\endcsname{\let\PY@it=\textit\def\PY@tc##1{\textcolor[rgb]{0.73,0.13,0.13}{##1}}}
\expandafter\def\csname PY@tok@si\endcsname{\let\PY@bf=\textbf\def\PY@tc##1{\textcolor[rgb]{0.73,0.40,0.53}{##1}}}
\expandafter\def\csname PY@tok@se\endcsname{\let\PY@bf=\textbf\def\PY@tc##1{\textcolor[rgb]{0.73,0.40,0.13}{##1}}}
\expandafter\def\csname PY@tok@sr\endcsname{\def\PY@tc##1{\textcolor[rgb]{0.73,0.40,0.53}{##1}}}
\expandafter\def\csname PY@tok@ss\endcsname{\def\PY@tc##1{\textcolor[rgb]{0.10,0.09,0.49}{##1}}}
\expandafter\def\csname PY@tok@sx\endcsname{\def\PY@tc##1{\textcolor[rgb]{0.00,0.50,0.00}{##1}}}
\expandafter\def\csname PY@tok@m\endcsname{\def\PY@tc##1{\textcolor[rgb]{0.40,0.40,0.40}{##1}}}
\expandafter\def\csname PY@tok@gh\endcsname{\let\PY@bf=\textbf\def\PY@tc##1{\textcolor[rgb]{0.00,0.00,0.50}{##1}}}
\expandafter\def\csname PY@tok@gu\endcsname{\let\PY@bf=\textbf\def\PY@tc##1{\textcolor[rgb]{0.50,0.00,0.50}{##1}}}
\expandafter\def\csname PY@tok@gd\endcsname{\def\PY@tc##1{\textcolor[rgb]{0.63,0.00,0.00}{##1}}}
\expandafter\def\csname PY@tok@gi\endcsname{\def\PY@tc##1{\textcolor[rgb]{0.00,0.63,0.00}{##1}}}
\expandafter\def\csname PY@tok@gr\endcsname{\def\PY@tc##1{\textcolor[rgb]{1.00,0.00,0.00}{##1}}}
\expandafter\def\csname PY@tok@ge\endcsname{\let\PY@it=\textit}
\expandafter\def\csname PY@tok@gs\endcsname{\let\PY@bf=\textbf}
\expandafter\def\csname PY@tok@gp\endcsname{\let\PY@bf=\textbf\def\PY@tc##1{\textcolor[rgb]{0.00,0.00,0.50}{##1}}}
\expandafter\def\csname PY@tok@go\endcsname{\def\PY@tc##1{\textcolor[rgb]{0.53,0.53,0.53}{##1}}}
\expandafter\def\csname PY@tok@gt\endcsname{\def\PY@tc##1{\textcolor[rgb]{0.00,0.27,0.87}{##1}}}
\expandafter\def\csname PY@tok@err\endcsname{\def\PY@bc##1{\setlength{\fboxsep}{0pt}\fcolorbox[rgb]{1.00,0.00,0.00}{1,1,1}{\strut ##1}}}
\expandafter\def\csname PY@tok@kc\endcsname{\let\PY@bf=\textbf\def\PY@tc##1{\textcolor[rgb]{0.00,0.50,0.00}{##1}}}
\expandafter\def\csname PY@tok@kd\endcsname{\let\PY@bf=\textbf\def\PY@tc##1{\textcolor[rgb]{0.00,0.50,0.00}{##1}}}
\expandafter\def\csname PY@tok@kn\endcsname{\let\PY@bf=\textbf\def\PY@tc##1{\textcolor[rgb]{0.00,0.50,0.00}{##1}}}
\expandafter\def\csname PY@tok@kr\endcsname{\let\PY@bf=\textbf\def\PY@tc##1{\textcolor[rgb]{0.00,0.50,0.00}{##1}}}
\expandafter\def\csname PY@tok@bp\endcsname{\def\PY@tc##1{\textcolor[rgb]{0.00,0.50,0.00}{##1}}}
\expandafter\def\csname PY@tok@fm\endcsname{\def\PY@tc##1{\textcolor[rgb]{0.00,0.00,1.00}{##1}}}
\expandafter\def\csname PY@tok@vc\endcsname{\def\PY@tc##1{\textcolor[rgb]{0.10,0.09,0.49}{##1}}}
\expandafter\def\csname PY@tok@vg\endcsname{\def\PY@tc##1{\textcolor[rgb]{0.10,0.09,0.49}{##1}}}
\expandafter\def\csname PY@tok@vi\endcsname{\def\PY@tc##1{\textcolor[rgb]{0.10,0.09,0.49}{##1}}}
\expandafter\def\csname PY@tok@vm\endcsname{\def\PY@tc##1{\textcolor[rgb]{0.10,0.09,0.49}{##1}}}
\expandafter\def\csname PY@tok@sa\endcsname{\def\PY@tc##1{\textcolor[rgb]{0.73,0.13,0.13}{##1}}}
\expandafter\def\csname PY@tok@sb\endcsname{\def\PY@tc##1{\textcolor[rgb]{0.73,0.13,0.13}{##1}}}
\expandafter\def\csname PY@tok@sc\endcsname{\def\PY@tc##1{\textcolor[rgb]{0.73,0.13,0.13}{##1}}}
\expandafter\def\csname PY@tok@dl\endcsname{\def\PY@tc##1{\textcolor[rgb]{0.73,0.13,0.13}{##1}}}
\expandafter\def\csname PY@tok@s2\endcsname{\def\PY@tc##1{\textcolor[rgb]{0.73,0.13,0.13}{##1}}}
\expandafter\def\csname PY@tok@sh\endcsname{\def\PY@tc##1{\textcolor[rgb]{0.73,0.13,0.13}{##1}}}
\expandafter\def\csname PY@tok@s1\endcsname{\def\PY@tc##1{\textcolor[rgb]{0.73,0.13,0.13}{##1}}}
\expandafter\def\csname PY@tok@mb\endcsname{\def\PY@tc##1{\textcolor[rgb]{0.40,0.40,0.40}{##1}}}
\expandafter\def\csname PY@tok@mf\endcsname{\def\PY@tc##1{\textcolor[rgb]{0.40,0.40,0.40}{##1}}}
\expandafter\def\csname PY@tok@mh\endcsname{\def\PY@tc##1{\textcolor[rgb]{0.40,0.40,0.40}{##1}}}
\expandafter\def\csname PY@tok@mi\endcsname{\def\PY@tc##1{\textcolor[rgb]{0.40,0.40,0.40}{##1}}}
\expandafter\def\csname PY@tok@il\endcsname{\def\PY@tc##1{\textcolor[rgb]{0.40,0.40,0.40}{##1}}}
\expandafter\def\csname PY@tok@mo\endcsname{\def\PY@tc##1{\textcolor[rgb]{0.40,0.40,0.40}{##1}}}
\expandafter\def\csname PY@tok@ch\endcsname{\let\PY@it=\textit\def\PY@tc##1{\textcolor[rgb]{0.25,0.50,0.50}{##1}}}
\expandafter\def\csname PY@tok@cm\endcsname{\let\PY@it=\textit\def\PY@tc##1{\textcolor[rgb]{0.25,0.50,0.50}{##1}}}
\expandafter\def\csname PY@tok@cpf\endcsname{\let\PY@it=\textit\def\PY@tc##1{\textcolor[rgb]{0.25,0.50,0.50}{##1}}}
\expandafter\def\csname PY@tok@c1\endcsname{\let\PY@it=\textit\def\PY@tc##1{\textcolor[rgb]{0.25,0.50,0.50}{##1}}}
\expandafter\def\csname PY@tok@cs\endcsname{\let\PY@it=\textit\def\PY@tc##1{\textcolor[rgb]{0.25,0.50,0.50}{##1}}}

\def\PYZbs{\char`\\}
\def\PYZus{\char`\_}
\def\PYZob{\char`\{}
\def\PYZcb{\char`\}}
\def\PYZca{\char`\^}
\def\PYZam{\char`\&}
\def\PYZlt{\char`\<}
\def\PYZgt{\char`\>}
\def\PYZsh{\char`\#}
\def\PYZpc{\char`\%}
\def\PYZdl{\char`\$}
\def\PYZhy{\char`\-}
\def\PYZsq{\char`\'}
\def\PYZdq{\char`\"}
\def\PYZti{\char`\~}
% for compatibility with earlier versions
\def\PYZat{@}
\def\PYZlb{[}
\def\PYZrb{]}
\makeatother


    % Exact colors from NB
    \definecolor{incolor}{rgb}{0.0, 0.0, 0.5}
    \definecolor{outcolor}{rgb}{0.545, 0.0, 0.0}



    
    % Prevent overflowing lines due to hard-to-break entities
    \sloppy 
    % Setup hyperref package
    \hypersetup{
      breaklinks=true,  % so long urls are correctly broken across lines
      colorlinks=true,
      urlcolor=urlcolor,
      linkcolor=linkcolor,
      citecolor=citecolor,
      }
    % Slightly bigger margins than the latex defaults
    
    \geometry{verbose,tmargin=1in,bmargin=1in,lmargin=1in,rmargin=1in}
    
    

    \begin{document}
    
    
    \maketitle
    
    

    
    \hypertarget{nauxefve-symbolic-automatic-differentiation-in-julia}{%
\section{Naïve Symbolic Automatic Differentiation in
Julia}\label{nauxefve-symbolic-automatic-differentiation-in-julia}}

In this document I'll show a quick naïve way of implementing a very
basic computer algebra system in Julia which can do symbolic derivatives
using automatic differentiation. I'll assume a basic knowldege of
working with Julia expressions.

Before we discuss derivatives, lets quickly build a way to do some basic
symbolic math.

First off, we need a data type for symbols and symbolic expressions. The
easiest choice is to use the built in \texttt{Symbol} and \texttt{Expr}
types, but if we were to to turn this code into a package people would
yell that defining new methods on functions from \texttt{Base} on types
from \texttt{Base} is heresy.

Nonetheless, I don't feel like building my own versions of
\texttt{Symbol} and \texttt{Expr} right now so I'll just make aliases
for those types called \texttt{Sym} and \texttt{SymExpr} and make all my
methods in terms of those so that later if I want to avoid type heresy,
I just have to change the definitions of \texttt{Sym} and
\texttt{SymExpr} and don't have to worry about all the methods I define.

    \begin{Verbatim}[commandchars=\\\{\}]
{\color{incolor}In [{\color{incolor}1}]:} \PY{n}{Sym} \PY{o}{=} \PY{k+kt}{Symbol}
        \PY{n}{SymExpr} \PY{o}{=} \PY{k+kt}{Expr}
        
        \PY{n}{symExpr}\PY{p}{(}\PY{n}{x}\PY{p}{)} \PY{o}{=} \PY{n}{x}
        \PY{n}{mathy} \PY{o}{=} \PY{k+kt}{Union}\PY{p}{\PYZob{}}\PY{n}{Sym}\PY{p}{,}\PY{n}{SymExpr}\PY{p}{,}\PY{k+kt}{Number}\PY{p}{\PYZcb{}}\PY{p}{;}
\end{Verbatim}


    Okay, now we have \texttt{mathy} which is just the union of
\texttt{Sym}, \texttt{SymExpr} and \texttt{Number} so we can define
methods for some of the standard mathematical functions

    \begin{Verbatim}[commandchars=\\\{\}]
{\color{incolor}In [{\color{incolor}27}]:} \PY{k}{function} \PY{n}{Base}\PY{o}{.}\PY{o}{:}\PY{o}{+}\PY{p}{(}\PY{n}{x}\PY{o}{::}\PY{n}{mathy}\PY{p}{,} \PY{n}{y}\PY{o}{::}\PY{n}{mathy}\PY{p}{)}
             \PY{k}{if} \PY{n}{x} \PY{o}{==} \PY{n}{y}
                 \PY{n}{symExpr}\PY{p}{(}\PY{o}{:}\PY{p}{(}\PY{l+m+mi}{2}\PY{o}{*}\PY{o}{\PYZdl{}}\PY{n}{x}\PY{p}{)}\PY{p}{)}
             \PY{k}{elseif} \PY{n}{x} \PY{o}{==} \PY{o}{\PYZhy{}}\PY{n}{y}
                 \PY{l+m+mi}{0}
             \PY{k}{elseif} \PY{n}{x} \PY{o}{==} \PY{l+m+mi}{0}
                 \PY{n}{y}
             \PY{k}{elseif} \PY{n}{y} \PY{o}{==} \PY{l+m+mi}{0}
                 \PY{n}{x}
             \PY{k}{else}
                 \PY{n}{symExpr}\PY{p}{(}\PY{o}{:}\PY{p}{(}\PY{o}{\PYZdl{}}\PY{n}{x}\PY{o}{+}\PY{o}{\PYZdl{}}\PY{n}{y}\PY{p}{)}\PY{p}{)}
             \PY{k}{end}
         \PY{k}{end}
         
         \PY{n}{Base}\PY{o}{.}\PY{o}{:}\PY{o}{+}\PY{p}{(}\PY{n}{x}\PY{o}{::}\PY{n}{mathy}\PY{p}{)} \PY{o}{=} \PY{n}{x}
\end{Verbatim}


    The with these methods on the \texttt{+} operator, we can do things like

\begin{Shaded}
\begin{Highlighting}[]
\NormalTok{julia> :x + :y}
\NormalTok{:(x + y)}

\NormalTok{julia> :x - :y}
\NormalTok{:(x - y)}

\NormalTok{julia> :x + :x}
\NormalTok{:(}\FloatTok{2}\NormalTok{x)}

\NormalTok{julia> +:x}
\NormalTok{:x}
\end{Highlighting}
\end{Shaded}

Notice that in the final \texttt{else} statement, I use
\texttt{symExpr(:(\$x+\$y))} which at this point just returns
\texttt{:(\$x+\$y)}, ie. \texttt{symExpr} is just an identiy function on
\texttt{Expr} types. Later, if we make our own \texttt{SymExpr} type to
avoide type piracy, then we just need to modify \texttt{symExpr} to
construct an object of type \texttt{SymExpr} instead of \texttt{Expr}
and we won't need to modify any of our arithmetic functions!

Now we can go and do the same for the subtration operator

    \begin{Verbatim}[commandchars=\\\{\}]
{\color{incolor}In [{\color{incolor}20}]:} \PY{k}{function} \PY{n}{Base}\PY{o}{.}\PY{o}{:}\PY{o}{\PYZhy{}}\PY{p}{(}\PY{n}{x}\PY{o}{::}\PY{n}{mathy}\PY{p}{,} \PY{n}{y}\PY{o}{::}\PY{n}{mathy}\PY{p}{)}
             \PY{k}{if} \PY{n}{x} \PY{o}{==} \PY{n}{y}
                 \PY{l+m+mi}{0}
             \PY{k}{elseif} \PY{n}{x} \PY{o}{==} \PY{o}{\PYZhy{}}\PY{n}{y}
                 \PY{n}{symExpr}\PY{p}{(}\PY{o}{:}\PY{p}{(}\PY{l+m+mi}{2}\PY{o}{\PYZdl{}}\PY{n}{x}\PY{p}{)}\PY{p}{)}
             \PY{k}{elseif} \PY{n}{x} \PY{o}{==} \PY{l+m+mi}{0}
                 \PY{o}{\PYZhy{}}\PY{n}{y}
             \PY{k}{elseif} \PY{n}{y} \PY{o}{==} \PY{l+m+mi}{0}
                 \PY{n}{x}
             \PY{k}{else}
                 \PY{n}{symExpr}\PY{p}{(}\PY{o}{:}\PY{p}{(}\PY{o}{\PYZdl{}}\PY{n}{x}\PY{o}{\PYZhy{}}\PY{o}{\PYZdl{}}\PY{n}{y}\PY{p}{)}\PY{p}{)}
             \PY{k}{end}
         \PY{k}{end}
         
         \PY{k}{function} \PY{n}{Base}\PY{o}{.}\PY{o}{:}\PY{o}{\PYZhy{}}\PY{p}{(}\PY{n}{x}\PY{o}{::}\PY{n}{Sym}\PY{p}{)}
             \PY{n}{symExpr}\PY{p}{(}\PY{o}{:}\PY{p}{(}\PY{o}{\PYZhy{}}\PY{o}{\PYZdl{}}\PY{n}{x}\PY{p}{)}\PY{p}{)}
         \PY{k}{end}
         
         \PY{n}{isUnaryOperation}\PY{p}{(}\PY{n}{ex}\PY{o}{::}\PY{n}{SymExpr}\PY{p}{)} \PY{o}{=} \PY{n}{length}\PY{p}{(}\PY{n}{ex}\PY{o}{.}\PY{n}{args}\PY{p}{)} \PY{o}{==} \PY{l+m+mi}{2}
         \PY{n}{car}\PY{p}{(}\PY{n}{x}\PY{o}{::}\PY{n}{SymExpr}\PY{p}{)} \PY{o}{=} \PY{n}{x}\PY{o}{.}\PY{n}{args}\PY{p}{[}\PY{l+m+mi}{1}\PY{p}{]}
         
         \PY{k}{function} \PY{n}{Base}\PY{o}{.}\PY{o}{:}\PY{o}{\PYZhy{}}\PY{p}{(}\PY{n}{x}\PY{o}{::}\PY{n}{SymExpr}\PY{p}{)}
             \PY{k}{if} \PY{p}{(}\PY{n}{car}\PY{p}{(}\PY{n}{x}\PY{p}{)} \PY{o}{==} \PY{o}{:}\PY{o}{\PYZhy{}}\PY{p}{)} \PY{o}{\PYZam{}\PYZam{}} \PY{p}{(}\PY{n}{x} \PY{o}{|\PYZgt{}} \PY{n}{isUnaryOperation}\PY{p}{)}
                 \PY{n}{symExpr}\PY{p}{(}\PY{n}{x}\PY{o}{.}\PY{n}{args}\PY{p}{[}\PY{l+m+mi}{2}\PY{p}{]}\PY{p}{)}
             \PY{k}{else}
                 \PY{n}{symExpr}\PY{p}{(}\PY{o}{:}\PY{p}{(}\PY{o}{\PYZhy{}}\PY{o}{\PYZdl{}}\PY{n}{x}\PY{p}{)}\PY{p}{)}
             \PY{k}{end}
         \PY{k}{end}
\end{Verbatim}


    The first method defined above tells us how to do basic subtration

\begin{Shaded}
\begin{Highlighting}[]
\NormalTok{julia> :x - :y}
\NormalTok{:(x - y)}

\NormalTok{julia> :x - :x}
\FloatTok{0}
\end{Highlighting}
\end{Shaded}

and the second two tell us how to negate a single argument, ie.

\begin{Shaded}
\begin{Highlighting}[]
\NormalTok{julia> -:x}
\NormalTok{:(-}\FloatTok{1}\NormalTok{x)}

\NormalTok{julia> -(-(:x + }\FloatTok{1}\NormalTok{))}
\NormalTok{:(x + }\FloatTok{1}\NormalTok{)}
\end{Highlighting}
\end{Shaded}

Now we can go and do similar things for multiplication, division,
exponentitation and logarithms

    \begin{Verbatim}[commandchars=\\\{\}]
{\color{incolor}In [{\color{incolor}30}]:} \PY{k}{function} \PY{n}{Base}\PY{o}{.}\PY{o}{:}\PY{o}{*}\PY{p}{(}\PY{n}{x}\PY{o}{::}\PY{n}{mathy}\PY{p}{,}\PY{n}{y}\PY{o}{::}\PY{n}{mathy}\PY{p}{)}
             \PY{k}{if} \PY{n}{x} \PY{o}{==} \PY{n}{y}
                 \PY{n}{symExpr}\PY{p}{(}\PY{o}{:}\PY{p}{(}\PY{o}{\PYZdl{}}\PY{n}{x}\PY{o}{\PYZca{}}\PY{l+m+mi}{2}\PY{p}{)}\PY{p}{)}
             \PY{k}{elseif} \PY{n}{x} \PY{o}{==} \PY{o}{\PYZhy{}}\PY{n}{y}
                 \PY{n}{symExpr}\PY{p}{(}\PY{o}{:}\PY{p}{(}\PY{o}{\PYZhy{}}\PY{o}{\PYZdl{}}\PY{n}{x}\PY{o}{\PYZca{}}\PY{l+m+mi}{2}\PY{p}{)}\PY{p}{)}
             \PY{k}{elseif} \PY{n}{x} \PY{o}{==} \PY{l+m+mi}{1}
                 \PY{n}{y}
             \PY{k}{elseif} \PY{n}{y} \PY{o}{==} \PY{l+m+mi}{1}
                 \PY{n}{x}
             \PY{k}{elseif} \PY{p}{(}\PY{n}{x} \PY{o}{==} \PY{l+m+mi}{0}\PY{p}{)} \PY{o}{||} \PY{p}{(}\PY{n}{y} \PY{o}{==} \PY{l+m+mi}{0}\PY{p}{)}
                 \PY{l+m+mi}{0}
             \PY{k}{else}
                 \PY{n}{symExpr}\PY{p}{(}\PY{o}{:}\PY{p}{(}\PY{o}{\PYZdl{}}\PY{n}{x}\PY{o}{*}\PY{o}{\PYZdl{}}\PY{n}{y}\PY{p}{)}\PY{p}{)}
             \PY{k}{end}
         \PY{k}{end}
         
         \PY{k}{function} \PY{n}{Base}\PY{o}{.}\PY{o}{:}\PY{o}{/}\PY{p}{(}\PY{n}{x}\PY{o}{::}\PY{n}{mathy}\PY{p}{,} \PY{n}{y}\PY{o}{::}\PY{n}{mathy}\PY{p}{)}
             \PY{k}{if} \PY{n}{x} \PY{o}{==} \PY{n}{y}
                 \PY{l+m+mi}{1}
             \PY{k}{elseif} \PY{n}{x} \PY{o}{==} \PY{o}{\PYZhy{}}\PY{n}{y}
                 \PY{o}{\PYZhy{}}\PY{l+m+mi}{1}
             \PY{k}{elseif} \PY{n}{y} \PY{o}{==} \PY{l+m+mi}{1}
                 \PY{n}{x}
             \PY{k}{else}
                 \PY{n}{symExpr}\PY{p}{(}\PY{o}{:}\PY{p}{(}\PY{o}{\PYZdl{}}\PY{n}{x}\PY{o}{/}\PY{o}{\PYZdl{}}\PY{n}{y}\PY{p}{)}\PY{p}{)}
             \PY{k}{end}
         \PY{k}{end}
         
         \PY{k}{function} \PY{n}{Base}\PY{o}{.}\PY{o}{:}\PY{o}{\PYZca{}}\PY{p}{(}\PY{n}{x}\PY{o}{::}\PY{n}{mathy}\PY{p}{,} \PY{n}{y}\PY{o}{::}\PY{n}{mathy}\PY{p}{)}
             \PY{n}{symExpr}\PY{p}{(}\PY{o}{:}\PY{p}{(}\PY{o}{\PYZdl{}}\PY{n}{x}\PY{o}{\PYZca{}}\PY{o}{\PYZdl{}}\PY{n}{y}\PY{p}{)}\PY{p}{)}
         \PY{k}{end}
         
         \PY{n}{Base}\PY{o}{.}\PY{o}{:}\PY{o}{\PYZca{}}\PY{p}{(}\PY{n}{x}\PY{o}{::}\PY{n}{mathy}\PY{p}{,} \PY{n}{y}\PY{o}{::}\PY{k+kt}{Int}\PY{p}{)} \PY{o}{=} \PY{n}{y} \PY{o}{==} \PY{l+m+mi}{1} \PY{o}{?} \PY{n}{x} \PY{o}{:} \PY{n}{symExpr}\PY{p}{(}\PY{o}{:}\PY{p}{(}\PY{o}{\PYZdl{}}\PY{n}{x}\PY{o}{\PYZca{}}\PY{o}{\PYZdl{}}\PY{n}{y}\PY{p}{)}\PY{p}{)}
         
         \PY{n}{Base}\PY{o}{.}\PY{n}{log}\PY{p}{(}\PY{n}{x}\PY{o}{::}\PY{n}{mathy}\PY{p}{)} \PY{o}{=} \PY{n}{symExpr}\PY{p}{(}\PY{o}{:}\PY{p}{(}\PY{n}{log}\PY{p}{(}\PY{o}{\PYZdl{}}\PY{n}{x}\PY{p}{)}\PY{p}{)}\PY{p}{)}
\end{Verbatim}


    Now we can write down all sorts of complicated symbolic expressions

\begin{Shaded}
\begin{Highlighting}[]
\NormalTok{julia> log(:x^}\FloatTok{4}\NormalTok{ + x^}\FloatTok{4}\NormalTok{)/:y + :y^(}\FloatTok{5}\NormalTok{(:x))}
\NormalTok{:(log(}\FloatTok{2}\NormalTok{ * x ^ }\FloatTok{4}\NormalTok{) / y + y ^ (}\FloatTok{5}\NormalTok{x))}
\end{Highlighting}
\end{Shaded}

\hypertarget{exercise}{%
\subsubsection{Exercise:}\label{exercise}}

Try to follow the conventions used above to define symbolic versions of
the trigonemtric functions

\hypertarget{automatic-differentiation}{%
\subsection{Automatic Differentiation}\label{automatic-differentiation}}

Now we've built up an basic computer algebra system. A common use for
such systems is the computation of derivatives. The conventional way to
do this is by defining a set of rules for transforming an expression
into its symbolic derivative. This technique is usually known simply as
`symbolic differentiation'. A more interesting technique for achieving
the same end is called `automatic differentiation'.

Recall from first year calculus that the derivative of a function
\(f(x)\) is defined as

\[
f'(x) \equiv \lim_{\Delta x \rightarrow 0} \frac{f(x+\Delta x) - f(x)}{\Delta x}
\]

We can also recall Taylor's theorem which states that for a well-behaved
function \(f(x)\),

\[
f(x + \Delta x) = f(x) + f'(x) ~ \Delta x + \frac{1}{2}~ f''(x) ~ (\Delta x)^2 + \frac{1}{6}~ f'''(x) ~ (\Delta x)^3 + \mathcal{O}(\Delta x ^4)
\]

So now, lets imagine a \(\Delta x\) which is so small that
\(\Delta x^2\) is \(0\) and we'll call this a `differential' \(dx\).

So now Taylor's theorm simply states that

\[
f(x +  dx) = f(x) + f'(x) ~ dx
\]

where we can imagine \(dx\) as an infinitesimally small number. In
standard numerical differentiation, one uses a small floating point
number not far above the minimum precision of the computer as their
value for \(dx\). Alternatively, we can make use of Julia's type system
and define a new type which represents quantities of the form
\(x + dx\).

The idea here is actually very similar to the construction of complex
numbers. With complex numbers, one simply says suppose there was such a
number \(i\) such that \(i^2 = -1\) and then defines a type (or in
mathematical language, an algebra) \texttt{Complex} (\(\mathbb{C}\)) and
defines what basic functions like \texttt{+,\ -,\ *,\ /,\ \^{},\ log} do
when acting on objects of type \texttt{Complex} (elements of
\(\mathbb{C}\)).

Similarly, we want to define a new number \(\epsilon\) such that
\(\epsilon^2 = 0\). Hence, we can make a type \texttt{Differential} and
then define methods on the \texttt{+,\ -,\ *,\ /,\ \^{},\ log} functions
which will give them mathematically correct results.

So first we make a new type:

    \begin{Verbatim}[commandchars=\\\{\}]
{\color{incolor}In [{\color{incolor}32}]:} \PY{k}{type} \PY{n}{Differential}
             \PY{n}{finite}\PY{o}{::}\PY{n}{mathy}
             \PY{n}{differential}\PY{o}{::}\PY{n}{mathy}
         \PY{k}{end}
         
         \PY{n}{differential}\PY{p}{(}\PY{n}{x}\PY{p}{,}\PY{n}{y}\PY{p}{)} \PY{o}{=} \PY{n}{y} \PY{o}{==} \PY{l+m+mi}{0} \PY{o}{?} \PY{n}{x} \PY{o}{:} \PY{n}{Differential}\PY{p}{(}\PY{n}{x}\PY{p}{,}\PY{n}{y}\PY{p}{)} 
         
         \PY{k}{function} \PY{n}{Base}\PY{o}{.}\PY{n}{show}\PY{p}{(}\PY{n}{io}\PY{o}{::}\PY{k+kt}{IO}\PY{p}{,} \PY{n}{x}\PY{o}{::}\PY{n}{Differential}\PY{p}{)}
             \PY{k}{if} \PY{p}{(}\PY{n}{x}\PY{o}{.}\PY{n}{finite}\PY{o}{==}\PY{l+m+mi}{0}\PY{p}{)} \PY{o}{\PYZam{}\PYZam{}} \PY{p}{(}\PY{n}{x}\PY{o}{.}\PY{n}{differential}\PY{o}{==}\PY{l+m+mi}{0}\PY{p}{)}
                 \PY{n}{print}\PY{p}{(}\PY{n}{io}\PY{p}{,}\PY{l+s}{\PYZdq{}}\PY{l+s}{0}\PY{l+s}{\PYZdq{}}\PY{p}{)}
             \PY{k}{else}
                 \PY{n}{finiteStr} \PY{o}{=} \PY{p}{(}\PY{n}{x}\PY{o}{.}\PY{n}{finite} \PY{o}{==} \PY{l+m+mi}{0} \PY{o}{?} \PY{l+s}{\PYZdq{}}\PY{l+s}{\PYZdq{}} \PY{o}{:} \PY{l+s}{\PYZdq{}}\PY{l+s+si}{\PYZdl{}}\PY{p}{(}\PY{n}{x}\PY{o}{.}\PY{n}{finite}\PY{p}{)}\PY{l+s}{ }\PY{l+s}{+}\PY{l+s}{ }\PY{l+s}{\PYZdq{}}\PY{p}{)}
                 \PY{n}{diffStr} \PY{o}{=} \PY{p}{(}\PY{n}{x}\PY{o}{.}\PY{n}{differential} \PY{o}{==} \PY{l+m+mi}{0} \PY{o}{?} \PY{l+s}{\PYZdq{}}\PY{l+s}{\PYZdq{}} \PY{o}{:} 
                     \PY{n}{x}\PY{o}{.}\PY{n}{differential} \PY{o}{==} \PY{l+m+mi}{1} \PY{o}{?} \PY{l+s}{\PYZdq{}}\PY{l+s}{\PYZdq{}} \PY{o}{:} 
                     \PY{n}{x}\PY{o}{.}\PY{n}{differential} \PY{n}{isa} \PY{k+kt}{Expr} \PY{o}{?} \PY{l+s}{\PYZdq{}}\PY{l+s}{(}\PY{l+s+si}{\PYZdl{}}\PY{p}{(}\PY{n}{x}\PY{o}{.}\PY{n}{differential}\PY{p}{)}\PY{l+s}{)}\PY{l+s}{\PYZdq{}} \PY{o}{:} \PY{l+s}{\PYZdq{}}\PY{l+s+si}{\PYZdl{}}\PY{p}{(}\PY{n}{x}\PY{o}{.}\PY{n}{differential}\PY{p}{)}\PY{l+s}{\PYZdq{}}\PY{p}{)}
                 \PY{n}{print}\PY{p}{(}\PY{n}{io}\PY{p}{,}\PY{n}{finiteStr}\PY{o}{*}\PY{n}{diffStr}\PY{o}{*}\PY{l+s}{\PYZdq{}}\PY{l+s}{ϵ}\PY{l+s}{\PYZdq{}}\PY{p}{)}
             \PY{k}{end}
         \PY{k}{end}
         
         \PY{n}{differential}\PY{p}{(}\PY{o}{:}\PY{n}{x}\PY{p}{,}\PY{o}{:}\PY{n}{y}\PY{p}{)}
\end{Verbatim}


\begin{Verbatim}[commandchars=\\\{\}]
{\color{outcolor}Out[{\color{outcolor}32}]:} x + yϵ
\end{Verbatim}
            
    The last function shown above just tells Julia how we'd like it to
display functions of type \texttt{Differential}.

\begin{Shaded}
\begin{Highlighting}[]
\NormalTok{julia> differential(:x, :y)}
\NormalTok{x + yϵ}

\NormalTok{julia> differential(}\FloatTok{0}\NormalTok{, (:x^}\FloatTok{2}\NormalTok{ + :y))}
\NormalTok{(x^}\FloatTok{2}\NormalTok{ + y)ϵ}

\NormalTok{julia> differential(:x, }\FloatTok{1}\NormalTok{)}
\NormalTok{x + ϵ}
\end{Highlighting}
\end{Shaded}

Now we can define a new union type that includes \texttt{Differential}s
and some helper functions for extracting out the \texttt{finite} and
infinitesimal parts of a quantity.

    \begin{Verbatim}[commandchars=\\\{\}]
{\color{incolor}In [{\color{incolor}35}]:} \PY{n}{mathyDiff} \PY{o}{=} \PY{k+kt}{Union}\PY{p}{\PYZob{}}\PY{n}{mathy}\PY{p}{,}\PY{n}{Differential}\PY{p}{\PYZcb{}}
         
         \PY{n}{finitePart}\PY{p}{(}\PY{n}{x}\PY{o}{::}\PY{n}{mathyDiff}\PY{p}{)} \PY{o}{=} \PY{n}{x} \PY{n}{isa} \PY{n}{Differential} \PY{o}{?} \PY{n}{x}\PY{o}{.}\PY{n}{finite} \PY{o}{:} \PY{n}{x}
         \PY{n}{diffPart}\PY{p}{(}\PY{n}{x}\PY{o}{::}\PY{n}{mathyDiff}\PY{p}{)} \PY{o}{=} \PY{n}{x} \PY{n}{isa} \PY{n}{Differential} \PY{o}{?} \PY{n}{x}\PY{o}{.}\PY{n}{differential} \PY{o}{:} \PY{l+m+mi}{0}\PY{p}{;}
\end{Verbatim}


    To see these in action

\begin{Shaded}
\begin{Highlighting}[]
\NormalTok{julia> z = differential(:x, :y);}
\NormalTok{julia> finitePart(z)}
\NormalTok{:x}

\NormalTok{julia> diffPart(z)}
\NormalTok{:y}

\NormalTok{julia> finitePart(}\FloatTok{1.05}\NormalTok{)}
\FloatTok{1.05}

\NormalTok{julia> diffPart(}\FloatTok{1.05}\NormalTok{)}
\FloatTok{0}
\end{Highlighting}
\end{Shaded}

    Now recall that

\[f'(x) = {(f(x+ϵ) - f(x)) \over ϵ}~~.\]

We can then make a Julian derivative function

    \begin{Verbatim}[commandchars=\\\{\}]
{\color{incolor}In [{\color{incolor}73}]:} \PY{n}{D}\PY{p}{(}\PY{n}{f}\PY{o}{::}\PY{k+kt}{Function}\PY{p}{)} \PY{o}{=} \PY{n}{x} \PY{o}{\PYZhy{}}\PY{o}{\PYZgt{}} \PY{n}{diffPart}\PY{p}{(}\PY{n}{f}\PY{p}{(}\PY{n}{x} \PY{o}{+} \PY{n}{ϵ}\PY{p}{)}\PY{p}{)}\PY{p}{;}
\end{Verbatim}


    Now once Julia has methods for the standard mathematical functions so
that they know how to deal with \texttt{Differential}s, our function
\texttt{D} will perform automatic differentiation!

Addition and aubtration are straightforward, you simply add (subtract)
the finite parts and the differential parts together separately:

    \begin{Verbatim}[commandchars=\\\{\}]
{\color{incolor}In [{\color{incolor}74}]:} \PY{k}{function} \PY{n}{Base}\PY{o}{.}\PY{o}{:}\PY{o}{+}\PY{p}{(}\PY{n}{x}\PY{o}{::}\PY{n}{mathyDiff}\PY{p}{,}\PY{n}{y}\PY{o}{::}\PY{n}{mathyDiff}\PY{p}{)}
             \PY{n}{differential}\PY{p}{(}\PY{n}{finitePart}\PY{p}{(}\PY{n}{x}\PY{p}{)} \PY{o}{+} \PY{n}{finitePart}\PY{p}{(}\PY{n}{y}\PY{p}{)}\PY{p}{,} \PY{n}{diffPart}\PY{p}{(}\PY{n}{x}\PY{p}{)} \PY{o}{+} \PY{n}{diffPart}\PY{p}{(}\PY{n}{y}\PY{p}{)}\PY{p}{)}
         \PY{k}{end}
         
         \PY{k}{function} \PY{n}{Base}\PY{o}{.}\PY{o}{:}\PY{o}{\PYZhy{}}\PY{p}{(}\PY{n}{x}\PY{o}{::}\PY{n}{mathyDiff}\PY{p}{,}\PY{n}{y}\PY{o}{::}\PY{n}{mathyDiff}\PY{p}{)}
             \PY{n}{differential}\PY{p}{(}\PY{n}{finitePart}\PY{p}{(}\PY{n}{x}\PY{p}{)} \PY{o}{\PYZhy{}} \PY{n}{finitePart}\PY{p}{(}\PY{n}{y}\PY{p}{)}\PY{p}{,} \PY{n}{diffPart}\PY{p}{(}\PY{n}{x}\PY{p}{)} \PY{o}{\PYZhy{}} \PY{n}{diffPart}\PY{p}{(}\PY{n}{y}\PY{p}{)}\PY{p}{)}
         \PY{k}{end}
\end{Verbatim}


    Now for multiplication. This is pretty strightforward, we simply use the
rule that \(\epsilon^2 = 0\). So lets suppose we have two
\texttt{Differential} numbers, \(u = x + y~ \epsilon\) and
\(v = z + w~ \epsilon\).

\begin{align}
u * v &= (x + y~\epsilon)(z + w~\epsilon)\\
&= xz + (xw + yz)\epsilon + yw \epsilon^2\\
&= xz + (xw + yz)\epsilon
\end{align}

In Julia we can express this proceedure as

    \begin{Verbatim}[commandchars=\\\{\}]
{\color{incolor}In [{\color{incolor}78}]:} \PY{k}{function} \PY{n}{Base}\PY{o}{.}\PY{o}{:}\PY{o}{*}\PY{p}{(}\PY{n}{x}\PY{o}{::}\PY{n}{mathyDiff}\PY{p}{,} \PY{n}{y}\PY{o}{::}\PY{n}{mathyDiff}\PY{p}{)}
             \PY{n}{differential}\PY{p}{(}\PY{n}{finitePart}\PY{p}{(}\PY{n}{x}\PY{p}{)}\PY{o}{*}\PY{n}{finitePart}\PY{p}{(}\PY{n}{y}\PY{p}{)}\PY{p}{,} \PY{n}{finitePart}\PY{p}{(}\PY{n}{x}\PY{p}{)}\PY{o}{*}\PY{n}{diffPart}\PY{p}{(}\PY{n}{y}\PY{p}{)} \PY{o}{+} \PY{n}{diffPart}\PY{p}{(}\PY{n}{x}\PY{p}{)}\PY{o}{*}\PY{n}{finitePart}\PY{p}{(}\PY{n}{y}\PY{p}{)}\PY{p}{)}
         \PY{k}{end}
         
         \PY{n}{ϵ} \PY{o}{=} \PY{n}{differential}\PY{p}{(}\PY{l+m+mi}{0}\PY{p}{,}\PY{l+m+mi}{1}\PY{p}{)}\PY{p}{;}
\end{Verbatim}


    With that definition, Julia now knows how to take derivatives of
functions involving multiplication and we get the product rule for free!

\begin{Shaded}
\begin{Highlighting}[]
\NormalTok{julia> f(x) = }\FloatTok{2}\NormalTok{*x;}
\NormalTok{julia> g(x) = }\FloatTok{4}\NormalTok{*x;}
\NormalTok{julia> h(x) = f(x)*g(x);}

\NormalTok{julia> f(:x+ϵ)}
\FloatTok{2}\NormalTok{*x + }\FloatTok{2}\NormalTok{ϵ}

\NormalTok{julia> D(f)(:x)}
\FloatTok{2}

\NormalTok{julia> g(:x + ϵ)}
\FloatTok{4}\NormalTok{*x + ϵ}

\NormalTok{julia> D(g)(:x)}
\FloatTok{4}

\NormalTok{julia> h(:x + ϵ)}
\NormalTok{(}\FloatTok{2}\NormalTok{x) * (}\FloatTok{4}\NormalTok{x) + ((}\FloatTok{2}\NormalTok{x) * }\FloatTok{4}\NormalTok{ + }\FloatTok{2}\NormalTok{ * (}\FloatTok{4}\NormalTok{x))ϵ}

\NormalTok{julia> D(h)(:x)}
\NormalTok{:((}\FloatTok{2}\NormalTok{x) * }\FloatTok{4}\NormalTok{ + }\FloatTok{2}\NormalTok{ * (}\FloatTok{4}\NormalTok{x))}
\end{Highlighting}
\end{Shaded}

We can see here that our automatic derivative function has correctly
performed the chain rule (though the simplification of the result left a
little to be desired).

We can teach Julia how to use the quotient rule as follows:

    \begin{Verbatim}[commandchars=\\\{\}]
{\color{incolor}In [{\color{incolor}40}]:} \PY{k}{function} \PY{n}{Base}\PY{o}{.}\PY{o}{:}\PY{o}{/}\PY{p}{(}\PY{n}{x}\PY{o}{::}\PY{n}{mathyDiff}\PY{p}{,} \PY{n}{y}\PY{o}{::}\PY{n}{mathyDiff}\PY{p}{)}
             \PY{p}{(}\PY{n}{finitePart}\PY{p}{(}\PY{n}{x}\PY{p}{)}\PY{o}{/}\PY{n}{finitePart}\PY{p}{(}\PY{n}{y}\PY{p}{)} \PY{o}{+} \PY{n}{ϵ}\PY{o}{*}\PY{p}{(}\PY{n}{finitePart}\PY{p}{(}\PY{n}{x}\PY{p}{)}\PY{o}{*}\PY{n}{diffPart}\PY{p}{(}\PY{n}{y}\PY{p}{)}\PY{o}{/}\PY{n}{finitePart}\PY{p}{(}\PY{n}{y}\PY{p}{)}\PY{o}{\PYZca{}}\PY{l+m+mi}{2}\PY{p}{)}\PY{p}{)} \PY{o}{+} \PY{n}{ϵ}\PY{o}{*}\PY{p}{(}\PY{n}{diffPart}\PY{p}{(}\PY{n}{x}\PY{p}{)}\PY{o}{/}\PY{n}{finitePart}\PY{p}{(}\PY{n}{y}\PY{p}{)}\PY{p}{)}
         \PY{k}{end}
\end{Verbatim}


    This simply says that \[
\frac{x}{y + \epsilon} = {x\over y}+ \frac{x}{y^2}\epsilon
\]

Now we can do

\begin{Shaded}
\begin{Highlighting}[]
\NormalTok{julia> D(x -> }\FloatTok{1}\NormalTok{/x)(:x)}
\FloatTok{1}\NormalTok{/:x^}\FloatTok{2}
\end{Highlighting}
\end{Shaded}

Likewise, we can teach Julia how to deal with exponents

    \begin{Verbatim}[commandchars=\\\{\}]
{\color{incolor}In [{\color{incolor}97}]:} \PY{k}{function} \PY{n}{Base}\PY{o}{.}\PY{o}{:}\PY{o}{\PYZca{}}\PY{p}{(}\PY{n}{x}\PY{o}{::}\PY{n}{Differential}\PY{p}{,} \PY{n}{y}\PY{o}{::}\PY{n}{mathy}\PY{p}{)}
             \PY{n}{finitePart}\PY{p}{(}\PY{n}{x}\PY{p}{)}\PY{o}{\PYZca{}}\PY{n}{y} \PY{o}{+} \PY{n}{y}\PY{o}{*}\PY{n}{finitePart}\PY{p}{(}\PY{n}{x}\PY{p}{)}\PY{o}{\PYZca{}}\PY{p}{(}\PY{n}{y}\PY{o}{\PYZhy{}}\PY{l+m+mi}{1}\PY{p}{)}\PY{o}{*}\PY{n}{diffPart}\PY{p}{(}\PY{n}{x}\PY{p}{)}\PY{o}{*}\PY{n}{ϵ}
         \PY{k}{end}
         
         \PY{k}{function} \PY{n}{Base}\PY{o}{.}\PY{o}{:}\PY{o}{\PYZca{}}\PY{p}{(}\PY{n}{x}\PY{o}{::}\PY{n}{Differential}\PY{p}{,} \PY{n}{y}\PY{o}{::}\PY{k+kt}{Int}\PY{p}{)}
             \PY{n}{finitePart}\PY{p}{(}\PY{n}{x}\PY{p}{)}\PY{o}{\PYZca{}}\PY{n}{y} \PY{o}{+} \PY{n}{y}\PY{o}{*}\PY{n}{finitePart}\PY{p}{(}\PY{n}{x}\PY{p}{)}\PY{o}{\PYZca{}}\PY{p}{(}\PY{n}{y}\PY{o}{\PYZhy{}}\PY{l+m+mi}{1}\PY{p}{)}\PY{o}{*}\PY{n}{diffPart}\PY{p}{(}\PY{n}{x}\PY{p}{)}\PY{o}{*}\PY{n}{ϵ}
         \PY{k}{end}
            
         \PY{k}{function} \PY{n}{Base}\PY{o}{.}\PY{o}{:}\PY{o}{\PYZca{}}\PY{p}{(}\PY{n}{x}\PY{o}{::}\PY{n}{mathy}\PY{p}{,} \PY{n}{y}\PY{o}{::}\PY{n}{Differential}\PY{p}{)}
             \PY{n}{x}\PY{o}{\PYZca{}}\PY{n}{finitePart}\PY{p}{(}\PY{n}{y}\PY{p}{)} \PY{o}{+} \PY{n}{log}\PY{p}{(}\PY{n}{x}\PY{p}{)}\PY{o}{*}\PY{n}{x}\PY{o}{\PYZca{}}\PY{n}{finitePart}\PY{p}{(}\PY{n}{y}\PY{p}{)}\PY{o}{*}\PY{n}{diffPart}\PY{p}{(}\PY{n}{y}\PY{p}{)}\PY{o}{*}\PY{n}{ϵ}
         \PY{k}{end}
\end{Verbatim}


    Now we can take derivatives of exponential functions!

\begin{Shaded}
\begin{Highlighting}[]
\NormalTok{julia> D(x -> x^}\FloatTok{2}\NormalTok{)(:x)}
\NormalTok{:(}\FloatTok{2}\NormalTok{x)}


\NormalTok{julia> D(x -> }\FloatTok{4}\NormalTok{*x^}\FloatTok{5}\NormalTok{)(:x)}
\NormalTok{:(}\FloatTok{4}\NormalTok{ * (}\FloatTok{5}\NormalTok{ * x ^ }\FloatTok{4}\NormalTok{))}

\NormalTok{julia> D(x -> (}\FloatTok{1}\NormalTok{ + x)*x^-}\FloatTok{4}\NormalTok{ )(:x)}
\NormalTok{:((}\FloatTok{1}\NormalTok{ + x) * (-}\FloatTok{4}\NormalTok{ * x ^ -}\FloatTok{5}\NormalTok{) + x ^ -}\FloatTok{4}\NormalTok{)}

\NormalTok{D(x -> }\FloatTok{2}\NormalTok{^x)(:x)}
\NormalTok{:(}\FloatTok{0.6931471805599453}\NormalTok{ * }\FloatTok{2}\NormalTok{ ^ x)}

\NormalTok{D(x -> x*}\FloatTok{2}\NormalTok{^x)(:x)}
\NormalTok{:(x * (}\FloatTok{0.6931471805599453}\NormalTok{ * }\FloatTok{2}\NormalTok{ ^ x) + }\FloatTok{2}\NormalTok{ ^ x)}
\end{Highlighting}
\end{Shaded}

\hypertarget{exercise}{%
\subsubsection{Exercise:}\label{exercise}}

You may notice that I have left out a method allowing Julia to take
derivatives of the form

\begin{Shaded}
\begin{Highlighting}[]
\NormalTok{julia> D(x -> x^x)(:x)}

\NormalTok{MethodError: no method matching ^(::Differential, ::Differential)}
\NormalTok{Closest candidates are:}
\NormalTok{  ^(::Differential, ::}\DataTypeTok{Int64}\NormalTok{) at In[}\FloatTok{91}\NormalTok{]:}\FloatTok{6}
\NormalTok{  ^(::Differential, ::}\DataTypeTok{Union}\NormalTok{\{}\DataTypeTok{Expr}\NormalTok{, }\DataTypeTok{Number}\NormalTok{, }\DataTypeTok{Symbol}\NormalTok{\}) at In[}\FloatTok{91}\NormalTok{]:}\FloatTok{2}
\NormalTok{  ^(::}\DataTypeTok{Union}\NormalTok{\{}\DataTypeTok{Expr}\NormalTok{, }\DataTypeTok{Number}\NormalTok{, }\DataTypeTok{Symbol}\NormalTok{\}, ::Differential) at In[}\FloatTok{91}\NormalTok{]:}\FloatTok{10}
\NormalTok{  ...}

\NormalTok{Stacktrace:}
\NormalTok{ [}\FloatTok{1}\NormalTok{] (::}\CommentTok{##7#8\{##23#24\})(::Symbol) at ./In[73]:1}
\NormalTok{ [}\FloatTok{2}\NormalTok{] include_string(::}\DataTypeTok{String}\NormalTok{, ::}\DataTypeTok{String}\NormalTok{) at ./loading.jl:}\FloatTok{522}
\end{Highlighting}
\end{Shaded}

To make such a derivative work, one needs to generalize the definition
of \texttt{\^{}} to take two \texttt{Differential} arguments. This isn't
difficult but it is messy so its left as an exercise to the reader.

\begin{center}\rule{0.5\linewidth}{\linethickness}\end{center}

    Finally, knowing that \(\frac{d~log(x)}{dx} = {1 \over x}\), we can
teach \texttt{log} to take \texttt{Differential\ aguments}:

    \begin{Verbatim}[commandchars=\\\{\}]
{\color{incolor}In [{\color{incolor}96}]:} \PY{k}{function} \PY{n}{Base}\PY{o}{.}\PY{n}{log}\PY{p}{(}\PY{n}{x}\PY{o}{::}\PY{n}{Differential}\PY{p}{)}
             \PY{n}{log}\PY{p}{(}\PY{n}{finitePart}\PY{p}{(}\PY{n}{x}\PY{p}{)}\PY{p}{)} \PY{o}{+} \PY{n}{diffPart}\PY{p}{(}\PY{n}{x}\PY{p}{)}\PY{o}{/}\PY{n}{finitePart}\PY{p}{(}\PY{n}{x}\PY{p}{)}\PY{o}{*}\PY{n}{ϵ}
         \PY{k}{end}
\end{Verbatim}


    and we can check if this gives us the correct behaviour:

\begin{Shaded}
\begin{Highlighting}[]
\NormalTok{D(x -> log(x))(:x)}
\NormalTok{:(}\FloatTok{1}\NormalTok{ / x)}

\NormalTok{D(x -> log(x)^}\FloatTok{2}\NormalTok{ + }\FloatTok{4}\NormalTok{*x)(:x)}
\NormalTok{:((}\FloatTok{2}\NormalTok{ * log(x)) * (}\FloatTok{1}\NormalTok{ / x) + }\FloatTok{4}\NormalTok{)}
\end{Highlighting}
\end{Shaded}

Great! With a little know-how and perserverence we've made a
rudiementary symbolic math system and taught it how to take derivatives
in Julian style!

    \hypertarget{exercise}{%
\subsubsection{Exercise:}\label{exercise}}

If you did the earlier exerise of definiting \texttt{mathy} methods for
the standard trigonometric functions, I suggest you see if you can
define \texttt{Differential} methods for them so that you can take their
derivatives.


    % Add a bibliography block to the postdoc
    
    
    
    \end{document}
